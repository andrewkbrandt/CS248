\documentclass[12pt]{article}
% Include any special packages you might use.
\usepackage{amssymb,amsmath,amsthm,cool} %Provides extra symbols
\usepackage[height=9in,width=6.5in]{geometry} %Adjusts margins
\parindent=0pt % disables indentation
% The following commands set up the material that appears
% in the header.
\usepackage{fancyhdr}
\lhead{CS 248: Homework 2.3} %change to section
%\chead{\today}  % Uncomment this line so add date to title

\rhead{Andrew Brandt} %Replace with your name
\pagestyle{fancy}
% If you want, you can make new commands, e.g. the following
\def\R{\mathbb R} %Define some commonly used symbols
\def\Q{\mathbb Q}
\def\Z{\mathbb Z}
\def\N{\mathbb N}

\title{CS 248 Unit 2 Homework 3}


%%%% Main document starts here.
\begin{document}
\textbf{Assignment due April 19th at 11:59 p.m. \\ \\
}Complete the following problems. These problems are related to Chapters 11 in the Book of Proof. Found at: \\ https://www.people.vcu.edu/~rhammack/BookOfProof/Main.pdf \\ \\

{\bf Problem 1:} Consider the relation $R=\{(a,b), (a,c), (c,c), (b,b), (c,b), (b,c)\}$ on the set $A= \{a, b, c\}.$ Is $R$ reflexive, symmetric, and/or transitive? For each property, briefly explain your answer. \\

    \begin{itemize}

        \item Reflexive: The relation is not reflexive because a does not directly relate to a.
        \item Symmetric: The relation is not symmetric because not every pair has a exact reverse. For example, the pair (a,b) does not have a matching (b,a) pair.
        \item Transitive: The relation is transitve because a relates to c and c relates to b.
    \end{itemize}

{\bf Problem 2:} Let $S = \{a, b, c\}.$ Let relation $R = \{(a,b), ...\}$ be a relation on set $S$. For each of the following sets of properties, provide a complete list of ordered pairs in $R$.
\begin{enumerate}
    \item Reflexive, Symmetric, Not Transitive
    
        $R = \{(a,a),(b,b),(c,c),(a,b),(b,a)\}$

    \item Not Reflexive, Symmetric, Transitive
    
        $R = \{(a,a),(b,b),(a,b),(b,c),(c,a)\}$

    \item Reflexive, Not Symmetric, Transitive
    
        $R = \{(a,a),(b,b),(c,c),(a,b),(b,c),(c,a)\}$

    \item Not Reflexive, Not Symmetric, Transitive
    
        $\{(a,b),(b,c),(c,a)\}$

\end{enumerate} 


{\bf Problem 3:} Prove that the divides relation on the set $\Z$ is reflexive, transitive, and not symmetric. Note that the divides relation is denoted as $a|b$, which means $b=ka$ for some integer $k$. 

\begin{proof}

    For a relation to be reflexive, every element must relate to itself.

    We know that $a|b$ means $b=ka$ for some $k\in\Z$. 
    
    Assume $k=1$, so $b=(1)a$.

    Therfore the divides relation is reflexive.

    For a relation to be symmetric, every pair must have a matching pair in opposite order.

    So lets say $a|b$ and $b=ka$ for some $k\in\Z$. For this to be true, $b|a$ must also be true.

    $b|a$ mean $a=kb$ for some $k\in\Z$
    
    So $b=ka$ and $a=kb$. This cannot be true because solving for a in the first equation would get $b/k = a$. this does not agree with the second function

    therefore it is not symmetric.
    
    %just prove it is transitive



\end{proof}

\end{document}