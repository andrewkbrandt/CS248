\documentclass[12pt]{article}
% Include any special packages you might use.
\usepackage{amssymb,amsmath,amsthm,cool} %Provides extra symbols
\usepackage[height=9in,width=6.5in]{geometry} %Adjusts margins
\parindent=0pt % disables indentation
% The following commands set up the material that appears
% in the header.
\usepackage{fancyhdr}
\lhead{CS 248: Homework 2.2} %change to section
%\chead{\today}  % Uncomment this line so add date to title

\rhead{Andrew Brandt} %Replace with your name
\pagestyle{fancy}
% If you want, you can make new commands, e.g. the following
\def\R{\mathbb R} %Define some commonly used symbols
\def\Q{\mathbb Q}
\def\Z{\mathbb Z}
\def\N{\mathbb N}

\title{CS 248 Unit 2 Homework 2}


%%%% Main document starts here.
\begin{document}
\textbf{Assignment due April 5th at 11:59 p.m. \\ \\
}Complete the following problems. These problems are related to Chapters 12 in the Book of Proof. Found at: \\ https://www.people.vcu.edu/~rhammack/BookOfProof/Main.pdf \\ \\

{\bf Problem 1:} For each of the following functions, determine whether the function is injective, surjective, bijective or none. Also, order the functions in a manner such that if a function $f(x)$ is to the left of $g(x)$ in the ordering, then $f(x)$ is $O(g(x)).$ You do not need to prove any of your conclusions. Assume the domain and co-domain of each function is $\R$.\\

We say $f(x)$ is $\mathcal{O}(g(x))$ when f(x) is less than or
equal to g(x) to within some constant multiple c

%%%918532 1/x logx 2x x^2 x^4 2^x x^x 
%%%% delete this later ^^^^
\begin{itemize}

    \item Ordering = $e(x)$, $h(x)$, $c(x)$, $a(X)$, $f(x)$, $g(x)$, $b(x)$, $d(x)$

    \item $a(x) = 2x$
    
    $a(x)$ is $\mathcal{O}(2x)$
    
    Injective, Surjective and Bijective

    \item $b(x) = 2^x$
    
    $b(x)$ is $\mathcal{O}(2^x)$
    
    Injective, not Surjective and Bijective

    \item $c(x) = \log(x)$
    
    $c(x)$ is $\mathcal{O}()\log(x))$
    
    Injective, not Surjective and not Bijective

    \item $d(x) = x^x$
    
    not Injective, not Surjective and not Bijective

    \item $e(x) = \frac{1}{x}$
    
    Injective, Surjective and Bijective

    \item $f(x) = \frac{x^2}{5}$
    
    not Injective, not Surjective and not Bijective

    \item $g(x) = \frac{x^4-1}{7x+7}$
    
    Injective, Surjective and Bijective

    \item $h(x) = 918532$
    
    not Injective, Surjective and not Bijective

\end{itemize} 

\textit{For problems 2 through 7, prove each of the statements.} \\ 

{\bf Problem 2:} The function $f:\Z \rightarrow \N$, $f(n)= n^2$ is not surjective and not injective.\\

\begin{proof}

    Assume $a=b$, so $(f(a)=a^2)=(f(b)=b^2)$

    we can say that $a^2$ is equal to $b^2$

    now lets set $a=-1$ and $b=1$

    they both equal 1 yet are different values

    therefore the function is not injective.

    Now lets assume that $n=0$

    it follows that $f(0)=0^2$ which is 0.

    zero is not in $\N$

    therefore, the function is not subjective.

\end{proof}

{\bf Problem 3:} The function $f:\N \rightarrow \Z$, $f(n)= 3n-17$ is injective.\\

\begin{proof}

    Assume $f(a)=f(b)$ for some $a,b\in\Z$

    It follows that $3a-17=3b-17$

    we can simplify to $3a=3b$ and then to $a=b$

    Therefore the function is injective.

\end{proof}

{\bf Problem 4:} The function $f:\R \rightarrow \Z$, $f(x)= \lfloor x \rfloor$ is surjective. \textit{Note that the floor function, denoted $\lfloor x \rfloor$, just rounds down. \\}

\begin{proof}

    let $a$ and $b$ be $\in \R$ and $<3$ and $>2$.

    $a\neq b$

    they will both round down to 2.

    therfore the function is surjective.


\end{proof}

{\bf Problem 5:} The function $f:\Z \rightarrow \Z$, $f(n)=-n + 4$, is a bijection. \textit{Hint: first find its inverse.} \\

\begin{proof}

    Lets start by finding the inverse of this $f$

    so $y=-x+4$, then switch x and y to get $x=-y+4$

    now rewrite with y on it's own side. $y=4-x$.

    so $f^{-1}(x)=4-x$ and $f(x)=-n+4$

    now look at $f(f{-1}(x))$ which is $-(4-x)+4$. This is equal to $x$

    They are inverses because this is equal to $x$.

    Therefore $f$ is bijective.

\end{proof}

{\bf Problem 6:} $\frac{x^3+17}{4x+5}$ is $O(3x^2)$. \\

\begin{proof}

    We say that $f(x)$ is $O(g(x))$ if there exists constants C and k such that $|f(x)\le C(g(x))$ for all $x>k$

    let $c=2$ and $k=1$, assume $x>k$.

    Then $|\frac{x^3+17}{4x+5}|=\frac{x^3+17}{4x+5}$ because the function will be positive for any $x>1$

    likewise, $|\frac{x^3+17}{4x+5}|\leq x\frac{^3+17x}{4x+5}$ because it is a bigger function.

    we can write this as $\frac{1}{4}x^2+17$

    then we can multiply the entire function by 24 and simplify to get $2\times 3x^2$

\end{proof}

{\bf Problem 7:} $n \log(n)$ is not $O(100 n)$. Assume log base 10.\\

\begin{proof}

    $f(x)$ is not $O(g(x))$ means for all constants C,k there exists an $x>k$ such that $|f(x)| > C|g(x)|$

    let C and k be arbitrary constants.

    Assume BYOC that $nlogn$ is $O(100n)$

    then there are C and k such that $0\le nlogn \le 100n$

    if a n is divided from both sides, we get $logn\le 100$

    this fails for  values of $x<10$

    therefore it is a contradiction.

\end{proof}

\end{document}