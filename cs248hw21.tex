\documentclass[12pt]{article}
% Include any special packages you might use.
\usepackage{amssymb,amsmath,amsthm,cool} %Provides extra symbols
\usepackage[height=9in,width=6.5in]{geometry} %Adjusts margins
\parindent=0pt % disables indentation
% The following commands set up the material that appears
% in the header.
\usepackage{fancyhdr}
\lhead{CS 248: Homework 2.1} %change to section
%\chead{\today}  % Uncomment this line so add date to title

\rhead{Andrew Brandt} %Replace with your name
\pagestyle{fancy}
% If you want, you can make new commands, e.g. the following
\def\R{\mathbb R} %Define some commonly used symbols
\def\Q{\mathbb Q}
\def\Z{\mathbb Z}
\def\N{\mathbb N}

\title{CS 248 Unit 2 Homework 1}


%%%% Main document starts here.
\begin{document}
Assignment due March 15th at 11:59 p.m. \\ \\
Complete the following problems from Chapters 4, 5, and 6 of Book of Proof. Found at: \\ https://www.people.vcu.edu/~rhammack/BookOfProof/Main.pdf \\ \\
Prove each of the following statements: \\ 

{\bf Problem 4.10} Suppose $a$ and $b$ are integers. If $a|b$, then $a|(3b^3-b^2+5b).$

{\bf Problem 4.14} If $n \in \Z$, then $5n^2+3n+7$ is odd.

{\bf Problem 4.26} Every odd integer is a difference of two squares. Example: $7 = 4^2-3^2.$

{\bf Problem 5.12} Suppose $a \in \Z$. If $a^2$ is not divisible by $4$, then $a$ is odd.

{\bf Problem 5.28} If $n \in \Z$, then $4 \not{|}  (n^2-3).$ 

{\bf Problem 6.8} Suppose $a,b,c \in \Z$. If $a^2+b^2=c^2$, then $a$ or $b$ is even \\


{\bf Proposition} Suppose $a$ and $b$ are integers. If $a|b$, then $a|(3b^3-b^2+5b).$

\begin{proof}

    we say $a|b$ for some $z\in\Z$ for $b=az$

    Therfore, we can write $3b^3-b^2+5b$ as $3(az)^3-(az)^2+5(az)$.

    Then factor out $a$ to get $a(3a^2b^3-az^2+5z)$.

    now a is multiplied to some constant and the product is $(3b^3-b^2+5b)$ 
    
    so the expression is true.

\end{proof}

{\bf Proposition} If $n\in\Z$, then $5n^2+3n+7$ is odd
\begin{proof}
    % an even plus an even is always an even
    % split it up into 3 individual parts
    % prove the first two are even
    % prove 7 is odd
    We say a number is odd if x=2k+1 for some integer k.

    assume 2 cases, n can be either even or odd.

    firstly, if n is odd.

    then we can say that $5n^2$ can be written as $5(2k+1)^2$

    this can be rewritten to $5(2k+1)(2k+1)$ then to

    $20k^2+10k+10K+2(2)+1$, then $2(10k^2+10k+2)+1$

    if we let $10k^2+10k+2 = j$, then $5n^2$ is odd because it can be written as $2j+1$

    next, look at $3n$, if n is odd, it can be written as $3(2k+1)$

    multiply out to get $6k+3$, rewrite this to $2(3k+1)+1$
    
    let $3k+1$ equal $h$, so $3n$ is odd because it can be written as $2h+1$

    Now that we know that $5n^2$ and $3n$ are both odd assuming that $n$ is even

    so adding these two terms together will result in a even number because

    $(2j+1)+(2h+1)$ or $2j+2h+2$, which can be written as $2(j+h)$ which is a even number.

    Now lets assume that n is even. An even number can be written as 2k for some interger k.

    looking at $5n^2$, we can write it as $5(2k)^2$, or $2(10k^2)$

    this number is even because if we write $10k^2$ as j, then it is $2j$

    also, $3n$ is even because we can write it as $3(2k)$, or $6k$

    which can then be written as $2(3k)$, if we make $3k$ h, then it would be $2h$

    Now that they are both even if n is even, adding them together will also get a even number.

    This is because, $2j+2h$, can be written as $2(j+h)$

    now that we know adding the first two terms will result in an even number wether n is even or odd

    we can prove 7 is odd because it can be written as $2(3)+1$,

    we know the first 2 terms added together will be even, so lets write them as 2j,

    so we have $2j+2(3)+1$, then a 2 can be factored out, $2(j+3)+1$.

    Therfore, $5n^2+3n+7$ is odd if $n\in Z$

\end{proof}

{\bf Proposition} Every odd integer is a difference of two squares.
\begin{proof}
    An integer is odd if it equals $2k+1$ for some integer k.

    now, lets look at 2 consecutive integers squared, written as k and k+1.

    $(k+1)^2 - k^2$, rewrite as $k^2+2k+1-k^2$. This is the same as $2k+1$

    thus, any prime can equal the difference of two squares.

\end{proof}

{\bf Proposition} Suppose $a\in \Z$. if $a^2$ is not divisible by 4, then $a$ is odd.
\begin{proof}

    We say that $a|b$ for some $z\in\Z$ for $b=az$

    lets assume that a is even by way of contradicton. $a=2k$ by definition of an even number.

    then $a^2$ would be equal to $4k^2$. $a^2$ is divisible by 4 because $a^2=4z$, with $z=k^2$.

    So for $a^2$ to be divisble by 4, it must be even.

\end{proof}

{\bf Proposition} If $n\in\Z$, then $4\not{|} (n^2-3)$
\begin{proof}
    By way of contradiction, assume $4|(n^2-3)$
    
    we say that $a|b$ for some $z\in\Z$ for $b=az$

    so $n^2-3=4z$.

    Then there are two cases to check, n is even or odd.

    first, assume n is even.

    so $(2k)^2-3=4z$, and rewrite it as $4k^2-3=4z$ and then solve for z.

    therefore $z=k^2-3/4$. this will not result in an integer

    likewise, assume n is odd.

    so $(2k+1)^2-3=4z$, and rewrite it as $4k^2+4k-2=4z$ and then solve for z.

    Therefore $z=k^2+4k-\frac{1}{2}$. This case will also not result in an integer.

\end{proof}

{\bf Proposiiton} Suppose $a,b,c \in \Z$. If $a^2+b^2=c^2$, then $a$ or $b$ is even
\begin{proof}

    By way of contradiction, lets assume that both a and b are odd.

    then we can write it as: $(2k+1)^2+(2j+1)^2=c^2$

    so $4(k^2+k+j^2+j)+2=c^2$

    then we know c can be either even or odd.

    assuming c is even, we get : $4h^2=4(k^2+k+j^2+j)+2$

    after dividing out the 4, there will be a fraction on one side.

    now lets assume that c is odd.

    so we get: $4(h^2+h)+1=4(k^2+k+j^2+j)+2=c^2$

    after dividing out the four, there will be unequal fractions on both sides.

    therfore it is not an integer for either case.

\end{proof}

\end{document}